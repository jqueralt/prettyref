% Usem la potència del paquet scrartcl de KOMA-script
\documentclass[fontsize=11pt,%
               paper=a4,%
               %BCOR=12mm,% marge per relligar
               %DIV=calc,%
               %twoside,%per defecte 1 cara
               %pagesize,%
               captions=tableheading,%
               numbers=noenddot,%
               parskip=full,%
               % toc=graduated%
               %toc=flat %índex amb ítems sense sagnar
               ]{scrartcl}  

% codificació
\usepackage[utf8]{inputenc}
\usepackage[T1]{fontenc}
\usepackage[catalan]{babel}                   % patrons de separació de paraules 
% tipus de lletra                                               
\usepackage[scaled=.88]{beramono}             % Bera-Monospace
\usepackage[scaled=.86]{berasans}             % Bera Sans-Serif
\usepackage[osf]{mathpazo}                    % Palatino com a font per defecte
\linespread{1.05}\selectfont                  % Palatino necessita més espai, aquí el 5%
\usepackage{ulem}                             % paquet per subratllar
%gràfics
\usepackage{eso-pic}                          % paquet per clavar imatges a pàgines
\usepackage{graphicx}                         % paquet per incloure imatges
\usepackage{color}                            % paquet per incloure colors
\usepackage{xcolor}                           % paquet per incloure colors
% per fer les caixes de color
\usepackage{tcolorbox}             % crea les caixes de color
\tcbuselibrary{skins}              % llibreria per fer caixes tipus beamer, crida a tikz
\usetikzlibrary{shadings,shadows}  % per usar la skin=beamer de tcolorbox
% paquets per taules
\usepackage{booktabs,array}                   % Taules agradables
\usepackage{ctable}  
\usepackage{colortbl}

\usepackage{lipsum}                % paquet per crear text de farciment
\usepackage{lastpage}              % paquet per posar el número d ela darrera pàgina

\usepackage{amsthm}                % paquet per millores matemàtiques, teoremes
\newtheorem{meuteo}{Teorema}       % definició del teorema




%%%%%%%%% paquet prettyref
% crida al paquet
\usepackage{prettyref}
% les etiquetes cal que portin un codi inicial per definir què són: sec: per secció, etc.
% definim els textos que apareixeran quan cridem a \prettyref:
\newrefformat{eq}{Equació \ref{#1} de la pàgina \pageref{#1}}
\newrefformat{lem}{Lema \ref{#1}}
\newrefformat{teo}{Teorema \ref{#1} de la pàgina \pageref{#1}}
%\newrefformat{cap}{Capítol \ref{#1}}
\newrefformat{sec}{Secció \ref{#1}}
\newrefformat{tau}{Taula \ref{#1} de la pàgina \pageref{#1}}
\newrefformat{fig}{Figura \ref{#1} de la pàgina \pageref{#1}}


% paquet pel formatatge de la pàgina: capçaleres i peus
\usepackage{scrpage2}
\clearscrheadfoot
\setlength{\headheight}{55pt}                           % espai extra per la imatge de la capçalera
\pagestyle{scrheadings}
\ohead[]{\includegraphics[scale=.25]{catalatex_logo.png}}
\ofoot[]{\thepage\ |\ \pageref{LastPage}}

% paquet per la configuració electrònica del PDF resultant
\definecolor{linkcolor}{rgb}{0,0,0.42}             % color de l'enllaç
\usepackage{hyperref}                              % configuració d'hyperref per pdf
  \hypersetup{%
    pdfauthor={(c) 2012 Joan Queralt},%
    pdftitle={Prova de prettyref},%
    pdfsubject={Millores de les referències creuades amb prettyref},%
    bookmarksopen=true, %finestreta de marcadors oberta quan s'obre el document
    linktocpage=true,
    urlcolor=linkcolor,
    citecolor=linkcolor,
    linkcolor=linkcolor,
    colorlinks=true,
  }
\def\meuteoautorefname{Teorema} % li diem a \autoref què posar on trobi el teorema ``meuteo``
  
% paquet per gestionar les extensions microtipogràfiques: protrusió dels caràcters i expansió dels tipus
\usepackage[stretch=10]{microtype}  

\author{Joan Queralt Gil}
\title{Prova de prettyref}
\date{\today}


\begin{document}
\maketitle

\tableofcontents
\newpage
%%%%%%%%%%%%%%%%%%%%%%%%%%%%%%%%%%%%%%%%%%%%%%%%%%%%%%
\section{Primera secció}\label{sec:primera}
\lipsum[1]

\begin{tcolorbox}[%
skin=beamer,                  %
width=\linewidth-2cm,         % escurça 2 cm per la dreta
fonttitle=\sffamily\bfseries, % tipus de lletra del títol
coltitle=black,               % color de la lletra del títol
colframe=yellow!50!red,       % color del marc i del fons de la zona superior
%colframe=red!30!yellow,      % color del marc i del fons de la zona superior
colback=white,                % color de fons de la zona inferior
title=Referència a taula]     % títol a la zona superior
Vegi's la \prettyref{tau:durada}. La referència creuada s'ha creat amb el comandament \verb+\prettyref{tau:durada}+.
\end{tcolorbox}

\lipsum[2-3]


\begin{meuteo}\label{teo:pitagores}
 Teorema de Pitàgores: La suma dels quadrarts dels catets és igual al quadrat de la hipotenusa:
 
 \[h^2 = c_1{^2} + c_2{^2} \]
\end{meuteo}

\lipsum[2-3]

\begin{tcolorbox}[%
skin=beamer,      
width=\linewidth-2cm,         % escurça 2 cm per la dreta
fonttitle=\sffamily\bfseries, % tipus de lletra del títol
coltitle=black,               % color de la lletra del títol
colframe=yellow!50!red,       % color del marc i del fons de la zona superior
%colframe=red!30!yellow,      % color del marc i del fons de la zona superior
colback=white,                % color de fons de la zona inferior
title=Referència a figura]     % títol a la zona superior
Vegi's la \prettyref{fig:qrcode}. La referència creuada s'ha creat amb el comandament \verb+\prettyref{fig:qrcode}+ 
\\I aqui amb \verb+\autoref{fig:qrcode}+: vegeu la \autoref{fig:qrcode}.
\end{tcolorbox}


\section{Segona secció}\label{sec:segona}
\lipsum[1-3].
\begin{equation}\label{eq:segongrau}
 x=\frac{-b\pm\sqrt{b^2-4ac}}{2a}
\end{equation}

\begin{tcolorbox}[%
skin=beamer,      
width=\linewidth-2cm,         % escurça 2 cm per la dreta
fonttitle=\sffamily\bfseries, % tipus de lletra del títol
coltitle=black,               % color de la lletra del títol
colframe=yellow!50!red,       % color del marc i del fons de la zona superior
%colframe=red!30!yellow,      % color del marc i del fons de la zona superior
colback=white,                % color de fons de la zona inferior
title=Referència a secció]     % títol a la zona superior
Anem a la \prettyref{sec:primera}. Aquesta referència creuada s'ha creat amb el comandament \verb+\prettyref{sec:primera}+
\end{tcolorbox}

\lipsum[2-3]

\section{Tercera secció}\label{sec:tercera}
\lipsum[1-3]

% Caixes de color
\begin{tcolorbox}[%
skin=beamer,      
width=\linewidth-2cm,         % escurça 2 cm per la dreta
fonttitle=\sffamily\bfseries, % tipus de lletra del títol
coltitle=black,               % color de la lletra del títol
colframe=yellow!50!red,       % color del marc i del fons de la zona superior
%colframe=red!30!yellow,      % color del marc i del fons de la zona superior
colback=white,                % color de fons de la zona inferior
title=Referència a teorema]     % títol a la zona superior
Vegi's el \prettyref{teo:pitagores}. Aquesta referència creuada s'ha creat amb el comandament \verb+\prettyref{teo:pitagores}+
\\I ara amb autoref: Vegi's el \autoref{teo:pitagores}. Aquesta referència creuada s'ha creat amb el comandament \verb+ \autoref{teo:pitagores}+
\end{tcolorbox}

\lipsum[2-3]

 \begin{table}[ht]
  \centering
  \begin{tabular}{llll}
    \toprule
    Tipus & Durada & gr/l & $K_c$ \\
    \midrule
    A & 10 h & 6.78 & 0.2345 \\
    B & 15 h & 3.49 & 0.1987 \\
    \bottomrule
  \end{tabular}
  \caption{Exemple de taula}
  \label{tau:durada}
\end{table}

\begin{tcolorbox}[%
skin=beamer,      
width=\linewidth-2cm,         % escurça 2 cm per la dreta
fonttitle=\sffamily\bfseries, % tipus de lletra del títol
coltitle=black,               % color de la lletra del títol
colframe=yellow!50!red,       % color del marc i del fons de la zona superior
%colframe=red!30!yellow,      % color del marc i del fons de la zona superior
colback=white,                % color de fons de la zona inferior
title=Referència a equació]     % títol a la zona superior
Veure l'\prettyref{eq:segongrau}. Aquesta referència creuada s'ha creat amb el comandament \verb+\prettyref{eq:segongrau}+
\end{tcolorbox}

\lipsum[2-4]

\begin{figure}[ht]
 \centering
 \includegraphics[scale=0.5,keepaspectratio=true]{./qrplanet.png}
 \label{fig:qrcode}
 \caption{Codi QR}
\end{figure}

\end{document}

%%%%%%%%%%%%%%%%%%%%%%%%%%%%%%%%%%%
% Caixes de color
\begin{tcolorbox}[%
width=\linewidth-2cm,         % escurça 2 cm per la dreta
fonttitle=\sffamily\bfseries, % tipus de lletra del títol
coltitle=black,               % color de la lletra del títol
colframe=yellow!50!red,       % color del marc i del fons de la zona superior
%colframe=red!30!yellow,      % color del marc i del fons de la zona superior
colback=white,                % color de fons de la zona inferior
title=Referència a taula]     % títol a la zona superior

\end{tcolorbox}
